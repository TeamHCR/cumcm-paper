\section{问题重述}

\subsection{问题背景}
在全球推进新能源发展的环境下,波浪能作为一种清洁无污染的可再生能源,既可降低石油开采量、节约资源,又可减少二氧化碳排放,是应对能源短缺以及环境问题,发展清洁能源、调整能源结构的重要选择之一。

我国波浪能资源丰富,可开发利用程度高,因此开发波浪能资源对解决我国资源问题具有重大意义。
同时,近些年来,波浪能利用已经成为全球范围的研究热点,波浪能发电装置种类繁多,层出不穷,各国学者都在积极探索高效转换能量的波浪装置和装置的分析方法。
本题所描述的就是一种摇荡式波能转换装置,它由一个外部「浮子」壳体和内部「振子」块体所构成,通过装置漂浮时浮子和振子之间产生的相对运动来驱动阻尼器做功,实现能量的捕获和转换。

\subsection{问题要求}

\subsubsection{问题一}

此题只考虑浮子在波浪中做垂荡运动的情况,要求建立浮子与振子的运动模型,根据题目所给出的参考值,分别针对直线阻尼器的阻尼系数的两种不同情况,计算浮子和振子在波浪激励力的作用下各个时间点的垂荡位移和速度。

\subsubsection{问题二}

此题仍只考虑浮子在波浪中只做垂荡运动的情况,同样分别针对上题的两种情况,建立确定最优阻尼系数的数学模型,使得输出功率最大,并计算出输出功率和最优阻尼系数的具体结果。

\subsubsection{问题三}

此题在垂荡运动基础上加入纵摇运动,规定扭转弹簧的扭矩与浮子和振子的相对角速度成正比,比例系数为旋转阻尼器的旋转阻尼系数。
考虑浮子同时做垂荡和纵摇运动时的情况,规定直线阻尼器和旋转阻尼器的阻尼系数为常量,要求建立浮子与振子的运动模型,计算浮子和振子在波浪激励力的作用下各个时间点时的垂荡位移和速度。

\subsubsection{问题四}

此题亦考虑浮子同时垂荡和纵摇运动,规定直线阻尼器和旋转阻尼器的阻尼系数为常量,要求建立确定两个最优阻尼系数的数学模型,使得输出功率最大,并计算出输出功率和最优阻尼系数的具体结果。


