\section{问题重述}

\subsection{问题背景}

项脊生曰:蜀清守丹穴,利甲天下,其后秦皇帝筑女怀清台。
刘玄德与曹操争天下,诸葛孔明起陇中,方二人之昧昧于一隅也,世何足以知之?
余区区处败屋中,方扬眉瞬目,谓有奇景;人知之者,其谓与坎井之蛙何异? 

余既为此志,后五年,吾妻来归。
时至轩中,从余问古事,或凭几学书。
吾妻归宁,述诸小妹语曰:“闻姊家有阁子,且何谓阁子也?”
其后六年,吾妻死,室坏不修。
其后二年,余久卧病无聊,乃使人复葺南阁子,其制稍异于前。
然自后余多在外,不常居。

\begin{equation}
    \vecmat{a}=\left(\int_a^b\e^xf(x)\dif{x}, 0\right)
\end{equation}

\textit{庭有枇杷树,吾妻死之年所手植也。今已亭亭如盖矣。 }

\subsection{问题要求}

\subsubsection{问题一}

豫章故郡,洪都新府。
星分翼轸,地接衡庐。
襟三江而带五湖,控蛮荆而引瓯越。
物华天宝,龙光射牛斗之墟;人杰地灵,徐孺下陈蕃之榻。
雄州雾列,俊彩星驰。
台隍枕夷夏之交,宾主尽东南之美。

\subsubsection{问题二}

都督阎公之雅望,棨戟遥临;宇文新州之懿范,襜帷暂驻。
十旬休暇,胜友如云。
千里逢迎,高朋满座。
腾蛟起凤,孟学士之词宗;紫电青霜,王将军之武库。家君作宰,路出名区。
童子何知?躬逢胜饯。 

\subsubsection{问题三}

时维九月,序属三秋。潦水尽而寒潭清,烟光凝而暮山紫。
俨骖𬴂于上路,访风景于崇阿。
临帝子之长洲,得仙人之旧馆。
层台耸翠,上出重霄;飞阁流丹,下临无地。
鹤汀凫渚,穷岛屿之萦回;桂殿兰宫,即冈峦之体势。 

