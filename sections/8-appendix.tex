\newpage
\section{程序源代码}

\subsection{Wolfram 程序源代码}

我们使用 Wolframe 语言编写了求解问题一至问题四的程序。
它们的源代码如下。
若要运行这些代码,请使用 Mathematica 13 或对应版本的 Wolfram Engine for Developers 软件输入下面的代码并运行。

此处的代码即是 \verb|final.nb| 文件中的输入命令集合。

\begin{lstlisting}[language=Mathematica,breaklines]
Clear["Global`*"];
问题 1
第 1 小问
m=2433;k=80000;c=10000;
f=6250;omega=1.4005;k1=1025*9.8*Pi;
c0=656.3616;m0=1335.535;M=4866;
stop=40/omega*2*Pi;
s=NDSolve[{m*x2''[t]==k(x1[t]-x2[t])+c(x1'[t]-x2'[t]),f*Cos[omega*t]==k(x1[t]-x2[t])+c(x1'[t]-x2'[t])+k1*x1[t]+c0*x1'[t]+m0*x1''[t]+M*x1''[t],x1[0]==x2[0]==x1'[0]==x2'[0]==0},{x1,x2},{t,0,Max[stop,100]}];
Plot[Evaluate[{x1[t],x2[t]}/.s],{t,0,Min[stop,100]},PlotLegends->Placed[{"x1","x2"},{1,0.5}],PlotStyle->{Directive[Blue,Thick],Directive[Orange,Dashed]}]
Plot[Evaluate[{x1[t]-x2[t],x1'[t]-x2'[t]}/.s],{t,0,Min[stop,100]},PlotLegends->Placed[{"\[CapitalDelta]x","\[CapitalDelta]v"},{1,0.5}]]

targets={10,20,40,60,100};
data=Table[{i,First[x1[i]/.s],First[x1'[i]/.s],First[x2[i]/.s],First[x2'[i]/.s]},{i,targets}];
Grid[data,Frame->All]
Export["~/cumcm/cumcm-code/P11-paper.xlsx",data]

Export["~/cumcm/cumcm-code/P11.xlsx",Flatten[Table[{i,x1[i],x1'[i],x2[i],x2'[i]}/.s,{i,Range[0,40/omega*2*Pi,0.2]}],1]]

第 (2) 小问
cv2[v_]:=10000*Sqrt[Abs[v]]
s=NDSolve[{m*x2''[t]==k(x1[t]-x2[t])+cv2[x2'[t]]*(x1'[t]-x2'[t]),f*Cos[omega*t]==k(x1[t]-x2[t])+cv2[x2'[t]]*(x1'[t]-x2'[t])+k1*x1[t]+c0*x1'[t]+m0*x1''[t]+M*x1''[t],x1[0]==x2[0]==x1'[0]==x2'[0]==0},{x1,x2},{t,0,Max[stop,100]}];
Plot[Evaluate[{x1[t],x2[t]}/.s],{t,0,Min[stop,100]},PlotLegends->Placed[{"x1","x2"},{1,0.5}],PlotStyle->{Directive[Green,Thick],Directive[Pink,Dashed]}]

targets={10,20,40,60,100};
data=Table[{i,First[x1[i]/.s],First[x1'[i]/.s],First[x2[i]/.s],First[x2'[i]/.s]},{i,targets}];
Grid[data,Frame->All]

Export["~/cumcm/cumcm-code/P12-paper.xlsx",data]
Plot[Evaluate[{x1[t]-x2[t],x1'[t]-x2'[t]}/.s],{t,0,Min[stop,100]},PlotLegends->Placed[{"\[CapitalDelta]x","\[CapitalDelta]v"},{1,0.5}]]
Export["~/cumcm/cumcm-code/P12.xlsx",Flatten[Table[{i,x1[i],x1'[i],x2[i],x2'[i]}/.s,{i,Range[0,40/omega*2*Pi,0.2]}],1]]

第 2 题
第 (1) 小问
Clear["Global`*"];
M=4866;m=2433;k=80000;c=10000;omega=2.2143;k1=1025*9.8*Pi;
f=4890;c0=167.8395;m0=1165.992;
start=110;stop=start+20*1/omega*2*Pi;dim=0.01;
P21[cm2_]:=(Sum[cm2*((x1'[i]-x2'[i])^2)*dim,{i,start,stop,dim}]/.NDSolve[{m*x2''[t]==k*(x1[t]-x2[t])+cm2*(x1'[t]-x2'[t]),f*Cos[omega*t]==k*(x1[t]-x2[t])+cm2*(x1'[t]-x2'[t])+k1*x1[t]+c0*x1'[t]+m0*x1''[t]+M*x1''[t],x1[0]==x2[0]==x1'[0]==x2'[0]==0},{x1,x2},{t,start,stop}])/(stop-start)
DrawP21[pStart_,pStop_,pStepNum_,name_]:=ListLinePlot[Table[{i,First[P21[i]]},{i,Range[pStart,pStop,(pStop-pStart)/pStepNum]}],PlotLegends->Placed[name,{1,0.5}]]
DrawP21[0,100000,100,"P(c)"]
DrawP21[37250,37270,100,"P(c)"]

第 (2) 小问
Clear["Global`*"];
M=4866;m=2433;k=80000;c=10000;omega=2.2143;k1=1025*9.8*Pi;
f=4890;c0=167.8395;m0=1165.992;
start=110;stop=start+20*1/omega*2*Pi;dim=0.01;
cmv2[v_,ck_,ce_]:=ck*Abs[v]^ce;
P22I[ck_,ce_]:=NDSolve[{m*x2''[t]==k*(x1[t]-x2[t])+cmv2[x1'[t]-x2'[t],ck,ce]*(x1'[t]-x2'[t]),f*Cos[omega*t]==k*(x1[t]-x2[t])+cmv2[x1'[t]-x2'[t],ck,ce]*(x1'[t]-x2'[t])+k1*x1[t]+c0*x1'[t]+m0*x1''[t]+M*x1''[t],x1[0]==x2[0]==x1'[0]==x2'[0]==0},{x1,x2},{t,start,stop}]
P22[ck_,ce_]:=(Sum[cmv2[x1'[i]-x2'[i],ck,ce+2]*dim,{i,start,stop,dim}]/.P22I[ck,ce])/(stop-start)
CalcP22[kStart_,kStop_,kStepNum_,eStart_,eStop_,eStepNum_]:=Parallelize[Table[{pe,ppk,First[P22[ppk,pe]]},{ppk,Range[kStart,kStop,(kStop-kStart)/kStepNum]},{pe,Range[eStart,eStop,(eStop-eStart)/eStepNum]}]]
DrawP22[data_]:=ListPlot3D[Flatten[data,1]]
s=CalcP22[0,100000,8,0,1,8];
ss=Table[{i[[1]],i[[2]]/10000,i[[3]]},{i,Flatten[s,1]}];
(*Grid[N[s],Frame->All]*)
ListPlot3D[ss]
ListContourPlot[ss]
(*ListDensityPlot[ss]*)

第 3 题
Clear["Global`*"];
(*AG长度*)
d=1.40792;
(*弹簧原长度*)
l0=0.5;
(*振子质量*)
m=2433;
(*弹簧劲度系数*)
k=80000;
(*直线阻尼器阻尼系数*)
c=10000;
(*小g*)
g=9.8;
(*弹簧旋转刚度*)
krot=250000;
(*旋转阻尼器阻尼系数*)
crot=1000;
(*浮子质量*)
M=4866;
(*垂荡兴波阻尼系数*)
c0=683.4558;
(*纵摇附加转动惯量*)
I0=7001.914;
(*纵摇激励力矩振幅*)
L=1690;
(*入射波浪频率*)
w=1.7152;
(*垂荡激励力振幅*)
f=3640;
(*浮子转动惯量*)
Ia=8399.2;

(*垂荡附加质量*)
m0=1028.876;
(*纵摇兴波阻尼力矩系数*)
cr0=654.3383;
(*纵摇静水恢复力矩系数*)
Mrec=8890.7;
(*海水密度*)
rho=1025;

(*垂荡静水恢复力(浮力)*)
Fj[t_]:=(5.2609+1.86007*Sec[th[t]]-Pi*zg[t]*Sec[th[t]])*rho*g;
(*激励力*)
Fwave[t_]:=f*Cos[w*t];
(*振子转动惯量*)
Ib[t_]:=m/12*(4*0.5^2-12*0.5*(0.5+l[t])+3*1^2+12*(0.5+l[t])^2)
(*纵摇激励力矩*)
Mwave[t_]:=L*Cos[w*t];
(*theta2=theta1+gamma*)
th2[t_]:=th[t]+ga[t];
(*P点x座标*)
xp[t_]:=d*Sin[th[t]]-l[t]*Sin[th2[t]];
(*P点z座标*)
zp[t_]:=zg[t]-d*Cos[th[t]]+l[t]*Cos[th2[t]];
(*PTO系统的力*)
Fpto[t_]:=-k*(l[t]-l0)-c*l'[t];
(*PTO系统的力矩*)
Mpto[t_]:=-krot*ga[t]-crot*ga'[t];

start=0;
stop=start+40/w*2*Pi;
group={m*xp''[t]==-Fpto[t]*Sin[th2[t]]+Fab[t]*Cos[th2[t]],m*zp''[t]==Fpto[t]*Cos[th2[t]]-m*g+Fab[t]*Sin[th2[t]],Ib[t]*th2''[t]==Mpto[t]+Fab[t]*l[t],M*zg''[t]==Fwave[t]+Fj[t]-M*g-Fpto[t]*Cos[th2[t]]-Fab[t]*Sin[th2[t]]-m0*zg''[t]-c0*zg'[t],Ia*th''[t]==Mwave[t]-Mpto[t]+Fab[t]*d*Cos[th[t]]*Cos[th2[t]]-Fab[t]*d*Sin[th[t]]*Sin[th2[t]]-I0*th''[t]-cr0*th'[t]-Mrec*th[t],th'[0]==th[0]==ga[0]==ga'[0]==l'[0]==zg'[0]==0,l[0]==l0-m/k,zg[0]==0};
s=NDSolve[group,{th,ga,zg,l},{t,start,Max[stop,100]}];
Plot[{th[t]/.s,ga[t]/.s},{t,start,Min[stop,100]},PlotLegends->Placed[{"\[Theta]","\[Gamma]"},{1,0.5}]]
Plot[{th'[t]/.s,ga'[t]/.s},{t,start,Min[stop,100]},PlotLegends->Placed[{"\[Theta]'","\[Gamma]'"},{1,0.5}]]
Plot[{l[t]/.s,zg[t]/.s},{t,start,Min[stop,100]},PlotLegends->Placed[{"l","Zg"},{1,0.5}]]
Plot[{l'[t]/.s,zg'[t]/.s},{t,start,Min[stop,100]},PlotLegends->Placed[{"l'","Zg'"},{1,0.5}]]
(*Plot[Fj[t]/.s,{t,0,stop},PlotLegends->{"F浮"}]*)

targets={10,20,40,60,100};
data=Table[{t,First[zg[t]/.s],First[zg'[t]/.s],First[th[t]/.s],First[th'[t]/.s],First[l[t]/.s],First[l'[t]/.s],First[ga[t]/.s],First[ga'[t]/.s]},{t,targets}];
Grid[data,Frame->All]
Export["~/cumcm/cumcm-code/P3-paper.xlsx",data]

dl[t_]:=l[t]-l0
data=Table[{t,First[ff[t]/.s]},{ff,{zg,zg',th,th',dl,l',ga,ga'}},{t,0,40/w*2*Pi,0.2}];
save=Table[{t,First[zg[t]/.s],First[zg'[t]/.s],First[th[t]/.s],First[th'[t]/.s],First[dl[t]/.s],First[l'[t]/.s],First[ga[t]/.s],First[ga'[t]/.s]},{t,0,40/w,0.2}];
ListLinePlot[data,PlotLegends->{Zg,Zg',theta,theta',dl,l',gamma,gamma'}]
Export["~/cumcm/cumcm-code/P3.xlsx",save]

第4题
Clear["Global`*"];
(*AG长度*)
d=1.40792;
(*弹簧原长度*)
l0=0.5;
(*振子质量*)
m=2433;
(*弹簧劲度系数*)
k=80000;
(*小g*)
g=9.8;
(*弹簧旋转刚度*)
krot=250000;
(*浮子质量*)
M=4866;
(*垂荡兴波阻尼系数*)
c0=528.5018;
(*纵摇附加转动惯量*)
I0=7142.493;
(*纵摇激励力矩振幅*)
L=2140;
(*入射波浪频率*)
w=1.9806;
(*垂荡激励力振幅*)
f=1760;
(*浮子转动惯量*)
Ia=8399.2;

(*垂荡附加质量*)
m0=1091.099;
(*纵摇兴波阻尼力矩系数*)
cr0=1655.909;
(*纵摇静水恢复力矩系数*)
Mrec=8890.7;
(*海水密度*)
rho=1025;

(*垂荡静水恢复力(浮力)*)
Fj[t_]:=(5.2609+1.86007*Sec[th[t]]-Pi*zg[t]*Sec[th[t]])*rho*g;
(*激励力*)
Fwave[t_]:=f*Cos[w*t];
(*振子转动惯量*)
(*Ib[t_]:=258.149+774.448l[t]+1548.9l[t]^2;*)
(*Ib[t_]:=MomentOfInertia[Cylinder[{{0,0,0},{0.5,0,0}},0.5],{l[t]+0.5,0,0},{0,0,1}]*)
Ib[t_]:=m/12*(4*0.5^2-12*0.5*(0.5+l[t])+3*1^2+12*(0.5+l[t])^2)
(*纵摇激励力矩*)
Mwave[t_]:=L*Cos[w*t];
(*theta2=theta1+gamma*)
th2[t_]:=th[t]+ga[t];
(*P点x座标*)
xp[t_]:=d*Sin[th[t]]-l[t]*Sin[th2[t]];
(*P点z座标*)
zp[t_]:=zg[t]-d*Cos[th[t]]+l[t]*Cos[th2[t]];
(*PTO系统的力*)
Fpto[t_,c_]:=-k*(l[t]-l0)-c*l'[t];
(*PTO系统的力矩*)
Mpto[t_,crot_]:=-krot*ga[t]-crot*ga'[t];

start=100;
stop=start+5/w*2*Pi;
dim=0.2;
P4[c_,crot_]:=((Sum[(c*(l'[t]^2)+crot*(ga'[t]^2))*dim,{t,start,stop,dim}]/.NDSolve[{m*xp''[t]==-Fpto[t,c]*Sin[th2[t]]+Fab[t]*Cos[th2[t]],m*zp''[t]==Fpto[t,c]*Cos[th2[t]]-m*g+Fab[t]*Sin[th2[t]],Ib[t]*th2''[t]==Mpto[t,crot]+Fab[t]*l[t],M*zg''[t]==Fwave[t]+Fj[t]-M*g-Fpto[t,c]*Cos[th2[t]]-Fab[t]*Sin[th2[t]]-m0*zg''[t]-c0*zg'[t],Ia*th''[t]==Mwave[t]-Mpto[t,crot]+Fab[t]*d*Cos[th[t]]*Cos[th2[t]]-Fab[t]*d*Sin[th[t]]*Sin[th2[t]]-I0*th''[t]-cr0*th'[t]-Mrec*th[t],th'[0]==th[0]==ga[0]==ga'[0]==l'[0]==zg'[0]==0,l[0]==l0-m/k,zg[0]==0},{th,ga,zg,l},{t,start,stop}])/(stop-start))
CalcP4[kStart_,kStop_,kStepNum_,eStart_,eStop_,eStepNum_]:=Parallelize[Table[{pe,ppk,First[P4[ppk,pe]]},{ppk,Range[kStart,kStop,(kStop-kStart)/kStepNum]},{pe,Range[eStart,eStop,(eStop-eStart)/eStepNum]}]]
s=CalcP4[0,100000,8,0,100000,8];

(*Grid[N[s],Frame->All]*)
ss=Flatten[s,1];
ListPlot3D[ss]
ListContourPlot[ss]
ListContourPlot[ss,PlotRange->{740,960}]    
\end{lstlisting}

\subsection{Python 程序源代码}

我们使用 Python 语言编写了求解问题二和问题四的程序。
它们的源代码如下。
若要运行这些代码,请使用 Python 3.10.4 及以上的 Python 运行程序。
此外,可能需要安装一些额外的 Python 模块。

此处的源代码即是支撑材料中的 \verb|P24.py| 文件。

\begin{lstlisting}[language=Python,breaklines=true]
import sys
import threading
import queue
import time
from wolframclient.evaluation import WolframLanguageSession
from wolframclient.language import wl, wlexpr
import matplotlib.pyplot as plt
import random
import math
from functools import reduce
# import nest_asyncio
# nest_asyncio.apply()
# from wolframclient.evaluation import parallel_evaluate

P21 = """
Clear["Global`*"];
M = 4866; m = 2433; k = 80000; c = 10000; omega = 2.2143; k1 = 1025 * 9.8 * Pi;
f = 4890; c0 = 167.8395; m0 = 1165.992;
start = 110; stop = start + 20 * 1 / omega*2*Pi; dim = 0.01;
P21[cm2_] := (Sum[cm2 * ((x1'[i] - x2'[i])^2) * dim, {i, start, stop, dim}] /. 
  NDSolve[{m*x2''[t]==k*(x1[t]-x2[t])+cm2*(x1'[t]-x2'[t]), 
    f*Cos[omega * t]==k*(x1[t]-x2[t])+cm2*(x1'[t]-x2'[t])+k1*x1[t]+c0*x1'[t]+m0*x1''[t]+M*x1''[t], 
    x1[0]==x2[0]==x1'[0]==x2'[0]==0},
  {x1, x2}, {t, start, stop}]) / (stop-start)
"""

P22 = """
Clear["Global`*"];
M = 4866; m = 2433; k = 80000; c = 10000; omega = 2.2143; k1 = 1025 * 9.8 * Pi;
f = 4890; c0 = 167.8395; m0 = 1165.992;
start = 110; stop = start + 20 * 1 / omega*2*Pi; dim = 0.01;
cmv2[v_, ck_, ce_] := ck * Abs[v]^ce;
P22I[ck_, ce_] := NDSolve[{m*x2''[t]==k*(x1[t]-x2[t])+cmv2[x1'[t]-x2'[t], ck, ce]*(x1'[t]-x2'[t]), 
    f*Cos[omega * t]==k*(x1[t]-x2[t])+cmv2[x1'[t]-x2'[t], ck, ce]*(x1'[t]-x2'[t])+k1*x1[t]+c0*x1'[t]+m0*x1''[t]+M*x1''[t], 
    x1[0]==x2[0]==x1'[0]==x2'[0]==0},
  {x1, x2}, {t, start, stop}]
P22[ck_, ce_] := (Sum[cmv2[x1'[i]-x2'[i], ck, ce+2] * dim, {i, start, stop, dim}] /. P22I[ck, ce]) / (stop-start)
"""

P4 = '''
Clear["Global`*"];
(* AG长度 *)
d = 1.40792;
(* 弹簧原长度 *)
l0 = 0.5; 
(* 振子质量 *)
m = 2433; 
(* 弹簧劲度系数 *)
k = 80000;
(* 小g *)
g = 9.8;
(* 弹簧旋转刚度 *)
krot = 250000;
(* 浮子质量 *)
M = 4866;
(* 垂荡兴波阻尼系数 *)
c0 = 528.5018;
(* 纵摇附加转动惯量 *)
I0 = 7142.493;
(* 纵摇激励力矩振幅 *)
L = 2140;
(* 入射波浪频率 *)
w = 1.9806;
(* 垂荡激励力振幅 *)
f = 1760;
(* 浮子转动惯量 *)
(* Ia = 300*4866/(7 \[Pi]+(Sqrt[41] \[Pi])/5)*(Integrate[1/0.8*2*Pi*(r+(17.6+8/75*(Sqrt[41]))/(7+(Sqrt[41])/5))*r^2,{r,-(17.6+8/75*(Sqrt[41]))/(7+(Sqrt[41])/5),-(17.6+8/75*(Sqrt[41]))/(7+(Sqrt[41])/5)+0.8}]+Integrate[2*Pi*r^2,{r,-(17.6+8/75*(Sqrt[41]))/(7+(Sqrt[41])/5)+0.8,-(17.6+8/75*(Sqrt[41]))/(7+(Sqrt[41])/5)+3.8}]+Integrate[2*Pi*(3.8-(17.6+8/75*(Sqrt[41]))/(7+(Sqrt[41])/5))^2+a^2,{a,0,0.5}]); *)
Ia = 8399.2;

(* 垂荡附加质量 *)
m0 = 1091.099;
(* 纵摇兴波阻尼力矩系数 *)
cr0 = 1655.909;
(* 纵摇静水恢复力矩系数 *)
Mrec = 8890.7;
(* 海水密度 *)
rho = 1025;

(* 垂荡静水恢复力(浮力) *)
Fj[t_] := (7.120975609756098 - Pi*zg[t]) * rho * g;
(* 激励力 *)
Fwave[t_] := f * Cos[w*t];
(* 振子转动惯量 *)
(* Ib[t_] := 258.149 + 774.448 l[t] + 1548.9 l[t]^2; *)
(* Ib[t_] := MomentOfInertia[Cylinder[{{0, 0, 0}, {0.5, 0, 0}}, 0.5], {l[t] + 0.5, 0, 0}, {0, 0, 1}] *)
Ib[t_] := m / 12 * (4*0.5^2-12*0.5*(0.5+l[t])+3*1^2+12*(0.5+l[t])^2)
(* 纵摇激励力矩 *)
Mwave[t_] := L * Cos[w*t];
(* theta2 = theta1 + gamma *)
th2[t_] := th[t] + ga[t]; 
(* P点x座标 *)
xp[t_] := d*Sin[th[t]] - l[t]*Sin[th2[t]]; 
(* P点z座标 *)
zp[t_] := zg[t] - d*Cos[th[t]] + l[t]*Cos[th2[t]]; 
(* PTO系统的力 *)
Fpto[t_, c_] := -k*(l[t] - l0) - c*l'[t];
(* PTO系统的力矩 *)
Mpto[t_, crot_] := -krot*ga[t] - crot*ga'[t];

start = 100;
stop = start + 5 / w * 2 * Pi;
dim = 0.2;

P4[c_, crot_] := ((Sum[(c * (l'[t]^2) + crot * (ga'[t]^2)) * dim, {t, start, stop, dim}] /. NDSolve[{
    m*xp''[t] == -Fpto[t,c]*Sin[th2[t]] + Fab[t]*Cos[th2[t]], 
    m*zp''[t] == Fpto[t,c]*Cos[th2[t]] - m*g + Fab[t]*Sin[th2[t]], 
    Ib[t]*th2''[t] == Mpto[t,crot] + Fab[t]*l[t],
    M*zg''[t] == Fwave[t] + Fj[t] - M*g - Fpto[t,c]*Cos[th2[t]] - Fab[t]*Sin[th2[t]] - m0*zg''[t] - c0*zg'[t],
    Ia*th''[t] == Mwave[t] - Mpto[t,crot] + Fab[t]*d*Cos[th[t]]*Cos[th2[t]] - Fab[t]*d*Sin[th[t]]*Sin[th2[t]] - I0*th''[t] - cr0*th'[t] - Mrec*th[t],
    th'[0] == th[0] == ga[0] == ga'[0] == l'[0] == zg'[0] == 0, l[0] == l0-m/k, zg[0] == 0}, {th, ga, zg, l}, {t, start, stop}]) / (stop - start))'''

queue_task = queue.Queue()
queue_res = queue.Queue()

random.seed(11451419198)
tot = 0
threads = []
thread_num = 16
thread_running = False
thread_cmd_name = "P22"
thread_command = P22

# use_parallelize = True
use_parallelize = False

def thread_run(command, name):
    # print("thread start:", threading.current_thread())
    session = WolframLanguageSession()
    session.evaluate(command)
    # print("test exec:", session.evaluate(wlexpr("P22[90000, 0.4]")))
    while True:
        try:
            try:
                task = queue_task.get(block=True, timeout=1)
                if task is None:
                    break
                # print("got", task)
                if thread_cmd_name == "P21":
                    text = f"First[{name}[{task[0]}]]"
                else:
                    text = f"First[{name}[{task[0]}, {task[1]}]]"
                try:
                    res = session.evaluate(wlexpr(text))
                    if not isinstance(res, float):
                        raise Exception("remote error")
                    # print("res", res, threading.current_thread())
                    queue_res.put(
                        {"id": task[2], "task": task, "res": res, "ok": True})
                except Exception as e:
                    print(e, threading.current_thread())
                    queue_res.put({'id': task[2], "ok": False})
                    continue
            except queue.Empty as e:
                # print("timeout")
                if thread_running:
                    continue
                else:
                    break
        except Exception as e:
            print(e, threading.current_thread())
            queue_res.put({'id': -1, "ok": False})
            continue
    session.stop()
    if thread_running:
        print("thread exit!", threading.current_thread())


def thread_pool_init(command=P22, name="P22"):
    global tot
    global threads
    global thread_running
    global thread_cmd_name
    global thread_command
    thread_running = True
    tot = 0
    thread_cmd_name = name
    thread_command = command
    if use_parallelize:
        threads = [threading.Thread(target=thread_run, daemon=True, args=(
            command, name)) for _ in range(thread_num)]
        [t.start() for t in threads]
    else:
        session_g.evaluate(command)
    print("thread_pool_init done")

def thread_pool_exit():
    global thread_running
    global threads
    thread_running = False
    if use_parallelize:
        [t.join() for t in threads]


def parallel_evaluate(exps, retry=0, **kwargs):
    if use_parallelize:
        exps = [(*exps[i], i) for i in range(len(exps))]
        [queue_task.put(i) for i in exps]
        results = [queue_res.get(block=True) for _ in range(len(exps))]
        failed = [it['id'] for it in results if not it['ok']]
        if len(failed) > 0:
            [queue_task.put(exps[failed[i]])
            for i in range(len(failed)) if failed[i] > 0]
            results2 = [queue_res.get(block=True, timeout=15)
                        for _ in range(len(failed))]
            results = [it for it in results if it['ok']]
            results.extend(results2)
        failed = [it['id'] for it in results if not it['ok']]
        # results = [queue_res.get(block=True, timeout=15) for _ in range(len(exps))]
        results.sort(key=lambda x: x['id'])
        res = [it['res'] for it in results if it['id'] >= 0 and it['ok']]
        if len(failed) > 0 or len(results) != len(exps):
            # time.sleep(2)
            print(res, failed)
            if retry >= 3:
                raise Exception("Has failed data!!!")
            else:
                return parallel_evaluate(exps, retry=retry + 1)
        return res
    else:
        # print("exps", exps)
        if thread_cmd_name == "P21":
            task = "{i[[1]],First[%s[i[[2]]]]}" % thread_cmd_name
            values = ["{%s,%f}" % (i, exps[i][0]) for i in range(len(exps))]
        else:
            task = "{i[[1]],First[%s[i[[2]],i[[3]]]]}" % thread_cmd_name
            values = ["{%s,%f,%f}" % (i, *exps[i]) for i in range(len(exps))]
        text = "Parallelize[Table[%s, {i, {%s}}]]" % (task, ','.join(values))
        # print(text)
        results = list(session_g.evaluate(wlexpr(text)))
        # print("result", results)
        results.sort(key=lambda x: x[0])
        res = [r[1] for r in results]
        # print([isinstance(r, float) for r in res])
        all_ok = sum([0 if isinstance(r, float) else 1 for r in res]) == 0
        if not all_ok:
            print(f"retry({retry}): {exps}")
            if retry >= 3:
                print(res)
                raise Exception("Has failed data!!!")
            else:
                return parallel_evaluate(exps, retry=retry + 1)
        # print("res", res)
        return res


def func(x, y):
    global tot
    tot += 1
    return (x, y)


ERR_RAND = 0

class SA:
    def __init__(self, func, iter=thread_num, T0=thread_num, Tf=10, alpha=0.99, x_range=None, y_range=None, sx=None, sy=None, overflow=5):
        self.func = func
        self.iter = iter  # 内循环迭代次数
        self.alpha = alpha  # 降温系数
        self.T0 = T0  # 初始温度T0
        self.Tf = Tf  # 温度终值Tf
        self.T = T0  # 当前温度
        self.x_range = [0, 100000] if x_range is None else x_range
        self.y_range = [0, 1.0] if y_range is None else y_range
        self.x = [random.random() * (self.x_range[1] - self.x_range[0]) +
                  self.x_range[0] for i in range(iter)]  # 随机生成iter个x的值
        self.y = [random.random() * (self.y_range[1] - self.y_range[0]) +
                  self.y_range[0] for i in range(iter)]  # 随机生成iter个y的值
        self.thread_num = iter
        self.most_best = []
        self.history = {'f': [], 'T': []}
        self.best_F = 1e10  # 记录最好的结果
        self.best_res = None
        self.sx = sx if sx is not None else self.x[0]
        self.sy = sy if sy is not None else self.y[0]
        self.count = 0
        self.time_start = 0
        self.time_stop = 0
        self.overflow = overflow

    def generate_new(self, x, y):  # 扰动产生新解的过程
        global ERR_RAND
        while True:
            dx = self.T / self.T0 * \
                (random.random() - random.random()) * \
                (self.x_range[1] - self.x_range[0])
            dy = self.T / self.T0 * \
                (random.random() - random.random()) * \
                (self.y_range[1] - self.y_range[0])
            x_new = x + dx
            y_new = y + dy
            if (self.x_range[0] <= x_new <= self.x_range[1]) and (self.y_range[0] <= y_new <= self.y_range[1]):
                ERR_RAND = 0
                break
            else:
                # print(f"(dx, dy) = ({dx}, {dy})", f"(x_new, y_new) = ({x_new}, {y_new})",
                #     (self.x_range[0] <= x_new <= self.x_range[1]), (self.y_range[0] <= y_new <= self.y_range[1]))
                ERR_RAND += 1
                # if ERR_RAND > 30:
                #     raise Exception("ERR_RAND overflow!")
        return x_new, y_new

    def generate_directions(self):
        x, y = self.sx, self.sy

        def d(r): return self.T / self.T0 * r * (random.random() - random.random())
        dirs = [[1, 0], [0, 1], [-1, 0], [0, -1]]
        ds = [(di[0] * d(self.x_range[1] - self.x_range[0]), di[1] *
               d(self.y_range[1] - self.y_range[0])) for di in dirs]

        def apply(dd):
            # print("dd", dd, "s", (self.sx, self.sy))
            x_new, y_new = dd[0] + x, dd[1] + y
            x_new = max(x_new, self.x_range[0])
            x_new = min(x_new, self.x_range[1])
            y_new = max(y_new, self.y_range[0])
            y_new = min(y_new, self.y_range[1])
            return (x_new, y_new)
        # res_dir = [apply(dd) for dd in ds]
        res_dir = []
        dscar = 0.3
        res_dir.extend([apply((dscar*d(self.x_range[1] - self.x_range[0]), dscar *
                       d(self.y_range[1] - self.y_range[0]))) for _ in range(thread_num-len(res_dir))])
        return res_dir

    # Metropolis准则
    # 1. 如果是更好的结果则接受
    # 2. 如果不是的话则概率接受
    def Metrospolis(self, f, f_new):
        if f_new <= f:
            # print(f"accept: {f} => {f_new}")
            return True
        else:
            p = math.exp((f - f_new) / self.T)
            return random.random() < p

    def best(self):  # 获取最优目标函数值
        f_list = []  # 每次迭代之后的值
        exps = [self.func(self.x[i], self.y[i]) for i in range(self.iter)]
        results_raw = parallel_evaluate(exps, max_evaluators=self.thread_num)
        f_list = [-i for i in results_raw]
        f_best = min(f_list)
        idx = f_list.index(f_best)
        if f_best < self.best_F:
            self.best_res = (self.x[idx], self.y[idx])
        return f_best, idx  # 在该温度下,迭代L次之后目标函数的最优解和最优解的下标

    def display(self):
        plt.plot(self.history['T'], self.history['f'])
        plt.title('SA')
        plt.xlabel('T')
        plt.ylabel('f')
        plt.gca().invert_xaxis()
        plt.show()

    def start_run_perf(self):
        self.count = 0
        self.time_start = time.time()

    def stop_run_perf(self):
        self.time_stop = time.time()
        if self.count > 0:
            print(
                f"speed = {self.count * 60 / (self.time_stop - self.time_start)} it/min, duration = {self.time_stop - self.time_start}, count = {self.count}")

    def run_random(self):
        self.start_run_perf()
        try:
            # 外循环迭代,当前温度小于终止温度的阈值
            while self.T > self.Tf:
                # 内循环迭代
                exps = [self.func(self.x[i], self.y[i]) for i in range(self.iter)]
                exps_all = [*exps]
                xy_new = [self.generate_new(self.x[i], self.y[i])
                        for i in range(self.iter)]
                exps_new = [self.func(*xy_new[i]) for i in range(self.iter)]
                exps_all = [*exps_all, *exps_new]
                results_all = parallel_evaluate(
                    exps_all, max_evaluators=self.thread_num)
                results_all = [-i for i in results_all]
                results_new, results_raw = results_all[self.iter:], results_all[:self.iter]
                assert(len(results_raw) == len(results_new))
                for i in range(self.iter):
                    f = results_raw[i]
                    f_new = results_new[i]
                    if self.Metrospolis(f, f_new):  # 判断是否接受新值
                        self.x[i] = xy_new[i][0]  # 如果接受新值,则把新值的x,y存入x数组和y数组
                        self.y[i] = xy_new[i][1]
                # 迭代L次记录在该温度下最优解
                ft, idx = self.best()
                self.history['f'].append(ft)
                self.history['T'].append(self.T)
                # 温度按照一定的比例下降(冷却)
                self.T = self.T * self.alpha
                print("Temp now:", self.T,
                    f"F={ft}, tot={tot}, x={self.x[idx]}, y={self.y[idx]}")
                self.count += 1

            # 得到最优解
            f_best, idx = self.best()
            if f_best < self.best_F:
                print(
                    f"F={f_best}, x={self.x[idx]}, y={self.y[idx]}, count={self.count}")
            else:
                print(
                    f"F={self.best_F}, x={self.best_res[0]}, y={self.best_res[1]}, count={self.count}")
        except KeyboardInterrupt:
            thread_pool_exit()
        self.stop_run_perf()

    def run_random_climb(self):
        self.start_run_perf()
        last_f = parallel_evaluate([self.func(self.sx, self.sy)])[0]
        print("start:", f"F={last_f}, x={self.sx}, y={self.sy}")
        over = 0
        try:
            while self.T > self.Tf:
                new_pos = self.generate_directions()
                new_exps = [self.func(*p) for p in new_pos]
                new_results = parallel_evaluate(new_exps)
                # if len(new_results) != 4:
                #     raise Exception("aaaaa 4 needed!")
                self.history['f'].append(last_f)
                self.history['T'].append(self.T)
                max_index = -1
                for i in range(len(new_results)):
                    if last_f < new_results[i]:
                        max_index = i
                        last_f = new_results[i]
                        self.sx = new_pos[i][0]
                        self.sy = new_pos[i][1]
                if max_index < 0:
                    # print(f"NO WAY! last={last_f}, new_results={new_results}")
                    if over > self.overflow:
                        print("Overflows!")
                        break
                    over += 1
                    pass
                else:
                    print("update:", self.T,
                          f"F={last_f}, tot={tot}, x={self.sx}, y={self.sy}, count={self.count}")
                    over = 0
                self.T = self.T * self.alpha
                self.count += 1

            print(f"x={self.sx}, y={self.sy}, F={last_f}")
        except KeyboardInterrupt:
            thread_pool_exit()
        print("Temp now:", self.T, f"F={last_f}, tot={tot}, x={self.sx}, y={self.sy}, count={self.count}")
        self.stop_run_perf()

session_g = WolframLanguageSession() if not use_parallelize else None

if __name__ == '__main__':
    # thread_cmd_name = "P4"
    thread_cmd_name = "P22"
    # thread_cmd_name = "P21"
    if len(sys.argv) > 1:
        if "22" in sys.argv[1]:
            thread_cmd_name = "P22"
        elif "21" in sys.argv[1]:
            thread_cmd_name = "P21"
        else:
            thread_cmd_name = "P4"
    print(f"thread_cmd_name = {thread_cmd_name}")
    if thread_cmd_name == "P4":
        if use_parallelize:
            thread_num = 6
        else:
            thread_num = 10
        random.seed(time.time())
        # update: 13.218697981369385 F=167.88943798950743, tot=321, x=44327.87533235402, y=30148.311347844916, count=19
        # update: 13.218697981369385 F=169.05912517083232, tot=321, x=43632.96648718373, y=32537.93339891905, count=19
        # update: 15.215840798399999 F=169.0599321072756, tot=97, x=43545.90864931983, y=32672.206619840814, count=5
        # Temp now: 5.140892816677745 F=169.06571859805382, tot=361, x=43408.15188934769, y=30062.39077757641, count=44
        # update: 14.913045566511839 F=167.8902517525808, tot=129, x=44492.630422162256, y=29644.097073419372, count=7
        # update: 11.483688521572397 F=167.89034175102677, tot=273, x=44493.76334705677, y=29418.730190797396, count=33
        # sa = SA(func, x_range=[30000, 50000], y_range=[25000, 40000], Tf=1e-1, sx=44493.76334705677, sy=29418.730190797396, overflow=300)
        sa = SA(func, x_range=[0, 100000], y_range=[0, 100000], Tf=1e-1, sx=44493.76334705677, sy=29418.730190797396, overflow=300)
        # sa = SA(func, x_range=[0, 100000], y_range=[0, 100000], Tf=1e-1, overflow=300)
        thread_pool_init(command=P4, name=thread_cmd_name)
    elif thread_cmd_name == "P21":
        # Temp now: 0.030233011657941858 F=230.5406624499153, tot=10001, x=37267.434515546855, count=624
        sa = SA(func, x_range=[0, 100000], y_range=[0, 1], Tf=1e-2, overflow=300, sx=37267.434515546855)
        thread_pool_init(command=P21, name=thread_cmd_name)
    else:  # 22
        # update: 9.392588510999751 F=231.1734673493091, tot=865, x=100000, y=0.4156291981028539, count=53
        sa = SA(func, x_range=[0, 100000], y_range=[0, 1], Tf=1e-2, sx=100000, sy=0.4156291981028539, overflow=300)
        thread_pool_init(command=P22, name=thread_cmd_name)
    sa.run_random_climb()
    sa.display()
    sys.exit(0) 
\end{lstlisting}