\section{模型假设}

\begin{enumerate}
    \item 兵者,国之大事,死生之地,存亡之道,不可不察也。
    故经之以五事,校之以计,而索其情:一曰道,二曰天,三曰地,四曰将,五曰法。
    \item 凡用兵之法,驰车千驷,革车千乘,带甲十万,千里馈粮,则内外之费,宾客之用,胶漆之材,车甲之奉,日费千金,然后十万之师举矣。
    其用战也,贵胜,久则钝兵挫锐,攻城则力屈,久暴师则国用不足。
    夫钝兵挫锐,屈力殚货,则诸侯乘其弊而起,虽有智者,不能善其后矣。
    故兵闻拙速,未睹巧之久也。
    夫兵久而国利者,未之有也。
    故不尽知用兵之害者,则不能尽知用兵之利也。
    \item 凡用兵之法,全国为上,破国次之;全军为上,破军次之;全旅为上,破旅次之;全卒为上,破卒次之;全伍为上,破伍次之。
    是故百战百胜,非善之善者也;不战而屈人之兵,善之善者也。
    \item 昔之善战者,先为不可胜,以待敌之可胜。
    不可胜在己,可胜在敌。
    故善战者,能为不可胜,不能使敌之必可胜。
    故曰:胜可知,而不可为。 
    \item 凡治众如治寡,分数是也;斗众如斗寡,形名是也;三军之众,可使必受敌而无败者,奇正是也;兵之所加,如以碫投卵者,虚实是也。 

\end{enumerate}