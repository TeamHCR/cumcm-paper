\section{符号说明}

在表 \ref{symbols} 中,我们约定了文中出现的各数学符号的含义和它们的单位。

\begin{table}[htbp]
    \centering
    \begin{tabular}{ccc}
        \toprule
        符号 & 含义 & 单位 \\
        \midrule
        $M$ & 浮子的质量 & $\mathrm{kg}$ \\
        $m$ & 振子的质量 & $\mathrm{kg}$ \\
        $m_0$ & 浮子受到的附加质量 & $\mathrm{kg}$ \\
        $g$ & 重力加速度 & $\mathrm{m}/\mathrm{s}^2$ \\
        $k$ & 弹簧的刚度系数 & $\mathrm{N}/\mathrm{m}$ \\
        $k_\text{rot}$ & 扭转弹簧刚度 & $\mathrm{N}\cdot\mathrm{m}$ \\
        $c$ & 阻尼器的阻尼系数 & $\mathrm{N}\cdot\mathrm{s}/\mathrm{m}$ \\
        $c_\text{rot}$ & 旋转阻尼器的阻尼系数 & $\mathrm{N}\cdot\mathrm{m}\cdot\mathrm{s}$ \\
        $c_0$ & 垂荡兴波阻尼系数 & $\mathrm{N}\cdot\mathrm{s}/\mathrm{m}$ \\ 
        $c_\text{r0}$ & 纵摇兴波阻尼系数 & $\mathrm{N}\cdot\mathrm{m}\cdot\mathrm{s}$ \\
        $I_A$ & 浮子的转动惯量 & $\mathrm{kg}\cdot\mathrm{m}^2$ \\
        $I_B$ & 振子的转动惯量 & $\mathrm{kg}\cdot\mathrm{m}^2$ \\
        $I_0$ & 纵摇附加转动惯量 & $\mathrm{kg}\cdot\mathrm{m}^2$ \\
        $M_\text{rec}$ & 静水恢复力矩系数 & $\mathrm{N}\cdot\mathrm{m}$ \\
        $F$ & 波浪激励力 & $\mathrm{N}$ \\
        $M_\text{wave}$ & 波浪激励力矩 & $\mathrm{N}\cdot\mathrm{m}$ \\
        $\omega$ & 波浪频率 & $\mathrm{s}^{-1}$ \\
        \bottomrule 
    \end{tabular}
    \caption{本文中出现的数学符号和它们的单位}
    \label{symbols}
\end{table}
