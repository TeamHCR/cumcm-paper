\section{模型检验与评价}


\subsection{模型存在的不足}

显然,我们在本文中所提出的模型存在着诸多不足,例如:

\begin{enumerate}
    \item 所建立的物理模型较为粗糙。一方面,我们选用的物理分析方法可能难以得到准确度较高的解。另一方面,我们对物理模型的计算较为理想化,如在确定浮子的转动中心时,我们是以其质心为转动中心来进行计算的。但事实上,浮子的运动状态必然比此情况复杂。
    \item 问题二和问题四中的算法存在改进空间。我们使用的「类」模拟退火算法,并没有进行严格的推理验证,我们所得到的最优情况也未必是全局的最优解。
    \item 模型并没有进行仿真或实际验证。在本领域的其他研究中,往往采用的是「理论建模+仿真验证」的研究思路,通过专业仿真软件乃至进行实验来验证模型的正确性,而我们对该问题的分析为纯理论的分析,因此正确性很难得到保证。
\end{enumerate}

\subsection{模型改进的方向}

在这些不足的基础上,我们也明确了此模型的改进方向:

\begin{enumerate}
    \item 使用更加精确的物理知识,建立更加完备的物理模型,综合考虑多种可能影响求解的因素,并将这些因素体现在物理模型的构建和计算过程中。
    \item 对优化算法进行调整,使其能更精确更完备地求解问题,同时取得性能上的提升。
    \item 在对物理模型进行理想化的简化时,要尽可能考虑到对结果有影响的变量,避免对物理模型进行简化导致计算结果偏差较大。同时,加入仿真流程,使模型的准确性能得到有效的验证。
\end{enumerate}