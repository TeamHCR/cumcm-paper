\section{模型建立与求解}

\subsection{问题一}

\subsubsection{模型的构建和分析}

在开始之前,我们需要先构建整个装置的物理模型。
以海平面为参考系。

当装置在水中静止时,对质量为 $m$ 的振子进行受力分析,则其受到的重力和受到弹簧弹力合力为零。
记以铅直向上为正方向,重力加速度为 $g$,记此时振子位移为 0,记此时弹簧压缩量为 $x_0$,弹簧的刚度为 $k$,则有 
\begin{equation}
    kx_0-mg=0 
    \label{oscilator}
\end{equation}

考虑在 $t$ 时刻,浮子位移 $x_1$,振子位移 $x_2$,正方向同前定义。
记阻尼器的阻尼系数为 $c$,对振子受力分析,如图 \ref{force-analysis} 左所示。我们有
\begin{equation}
    m\dfrac{\dif^2x_2}{\dif t^2}=k(x_1-x_2+x_0)+c(\dfrac{\dif{x_1}}{\dif{t}}-\dfrac{\dif{x_2}}{\dif{t}})-mg
\end{equation}
其中,$\dfrac{\dif^2 x_i}{\dif t^2}$ 即为瞬时加速度,$\dfrac{\dif x_i}{\dif t}$ 即为瞬时速度。

与此同时,我们对浮子进行受力分析,如图 \ref{force-analysis} 右所示。
记海水的兴波阻尼系数为 $c_0$,附加质量为 $m_0$,浮子质量为 $M$,波浪激励力为 $F$,有 
\begin{equation}
    M\dfrac{\dif^2x_1}{\dif t^2}=F+F_\text{浮}-Mg-k(x_1-x_2+x_0)-c(\dfrac{\dif{x_1}}{\dif{t}}-\dfrac{\dif{x_2}}{\dif{t}})-c_0\dfrac{\dif{x_1}}{\dif{t}}-m_0\dfrac{\dif^2 x_1}{\dif x_1^2}
\end{equation}

\begin{figure}[htbp]
    \centering
    \includegraphics[width=11cm]{fig/force-analysis.pdf}
    \caption{对振子和浮子进行受力分析。为制图方便,未画出完整的浮子壳体}
    \label{force-analysis}
\end{figure}

我们需要确定起始时装置的深度。
记装置静止时受到的总浮力为 $F_\text{浮总}$,水的密度为 $\rho$,装置在水中漂浮时排水体积为 $V_\text{排总}$,则有
\begin{equation}
    \left\{
    \begin{aligned}
        & F_\text{浮总}=\rho gV_\text{排总} \\
        & (m+M)g=F_\text{浮总}
    \end{aligned}
    \right.
    \label{floating-relationship}
\end{equation}
代入附件 4 的数据容易求得 $V_\text{排总}\approx7.121\,\mathrm{m}^3$。
另一方面,考虑浮子下方锥体的体积,有 $V_\text{锥}=\dfrac{1}3\uppi r^2h\approx0.837\,\mathrm{m}^3$。显然,水没过了整个锥体部分,水面在浮子的柱体部分。此时,浮子因在液面中上升下降所造成的浮力变化 $\Delta F_\text{浮}=\rho g\Delta V_{排}=\rho gA\Delta x_1$,其中 $\Delta x_1$ 为浮子没入液面深度的变化量,$A$ 是圆柱的横截面面积。
从而有
\begin{equation}
    F_\text{浮}=F_\text{浮总}+\Delta F_\text{浮}=F_\text{浮总}-\rho gAx_1
    \label{floating}
\end{equation}

联立 \eqref{oscilator}—\eqref{floating} 式,我们可以消去 $mg$、$Mg$、$F_\text{浮}$ 和 $x_0$ 几项,得到
\begin{equation}
    \left\{
    \begin{aligned}
        &  m\dfrac{\dif^2x_2}{\dif t^2}=k(x_1-x_2)+c(\dfrac{\dif{x_1}}{\dif{t}}-\dfrac{\dif{x_2}}{\dif{t}}) \\
        &  M\dfrac{\dif^2x_1}{\dif t^2}=F-k(x_1-x_2)-c(\dfrac{\dif{x_1}}{\dif{t}}-\dfrac{\dif{x_2}}{\dif{t}})-\rho gAx_1-c_0\dfrac{\dif{x_1}}{\dif{t}}-m_0\dfrac{\dif^2 x_1}{\dif t^2}
    \end{aligned}
    \right.
    \label{chui-dang}
\end{equation}
式 \eqref{chui-dang} 就是关于浮子和振子垂荡运动的微分方程组。
考虑装置启动前静止,$t=0$ 时,有初值
\begin{equation}
    \left\{
    \begin{aligned}
        & x_1=x_2=0 \\
        & \dfrac{\dif x_1}{\dif t}=\dfrac{\dif x_2}{\dif t}=0
    \end{aligned}
    \right.
\end{equation}
求解这个初值问题,我们就能解出此题。

\subsubsection{问题求解}

根据附件 3 和附件 4,我们查得式 \eqref{chui-dang} 中的各个常数值如表 \ref{consts-1} 所示。
\begin{table}[htbp]
    \centering
    \begin{tabular}{cc}
        \toprule
        符号 & 常数的值 \\
        \midrule
        $M$ & \N{4866}\,kg \\
        $m$ & \N{2433}\,kg \\
        $k$ & \N{250000}\,$\mathrm{N}\cdot\mathrm{m}$ \\
        $\rho$ & \N{1025}\,$\mathrm{kg}/\mathrm{m}^3$ \\
        $g$ & \N{9.8}\,$\mathrm{m}/\mathrm{s}^2$ \\
        $A$ & $\uppi\,\mathrm{m}^2$ \\
        $c_0$ & \N{656.3616}\,$\mathrm{N}\cdot\mathrm{s}/\mathrm{m}$ \\
        $m_0$ & \N{1335.535}\,kg \\
        \bottomrule
    \end{tabular}
    \caption{问题一适用的部分常量}
    \label{consts-1}
\end{table}

同时,对于波浪激励力 $F$ 我们有
\begin{equation}
    F=f\cos(\omega t)
\end{equation}
其中 $f=\N{6250}\,\mathrm{N}$,$\omega=\N{1.4005}\,\mathrm{s}^{-1}$。
而对于阻尼系数 $c$,有
\begin{equation}
    c=\left\{
        \begin{aligned}
            & \N{10000}\,\mathrm{N}\cdot\mathrm{s}/\mathrm{m}, \text{对于情况 1} \\
            & \N{10000}\sqrt{\left|\dfrac{\dif x_1}{\dif t}-\dfrac{\dif x_2}{\dif t}\right|}\,\mathrm{N}\cdot(\mathrm{s}/\mathrm{m})^{3/2}, \text{对于情况 2}
        \end{aligned}
    \right.
\end{equation}

我们使用计算机对这个二元二阶线性初值问题进行数值求解。
对于题设的两种情况,我们均设定求解起点和终点为 0\,s 和 180\,s,以 0.2\,s 为间隔进行求解,得到在这 180\,s(即前 40 个周期)中,振子和浮子各自的位移和速度,如图所示。

完整的求解结果,我们已经存放在 \verb|result1-1.xlsx| 和 \verb|result1-2.xlsx| 文件中。其中,10\,s,20\,s,40\,s,60\,s,100\,s 时的数据如表 \ref{answer-1-1} 和表 \ref{answer-1-2} 所示。
\begin{table}[htbp]
    \centering
    \begin{tabular}{ccccccc}
        \toprule
        & & 10\,s & 20\,s & 40\,s & 60\,s & 100\,s \\
        \midrule
        \multirow{2}{*}{浮子} & 速度/$\mathrm{m}\cdot\mathrm{s}^{-1}$ & 1 & 2 & 3 & 4 & 5 \\
        & 位移/$\mathrm{m}$ & 1 & 2 & 3 & 4 & 5 \\
        \multirow{2}{*}{振子} & 速度/$\mathrm{m}\cdot\mathrm{s}^{-1}$ & 1 & 2 & 3 & 4 & 5 \\
        & 位移/$\mathrm{m}$ & 1 & 2 & 3 & 4 & 5 \\
        \bottomrule
    \end{tabular}
    \caption{当 $c$ 为定值时,求解得到几个特殊时刻的浮子、振子位移和速度数据}
    \label{answer-1-1}
\end{table}

\begin{table}[htbp]
    \centering
    \begin{tabular}{ccccccc}
        \toprule
        & & 10\,s & 20\,s & 40\,s & 60\,s & 100\,s \\
        \midrule
        \multirow{2}{*}{浮子} & 速度/$\mathrm{m}\cdot\mathrm{s}^{-1}$ & 1 & 2 & 3 & 4 & 5 \\
        & 位移/$\mathrm{m}$ & 1 & 2 & 3 & 4 & 5 \\
        \multirow{2}{*}{振子} & 速度/$\mathrm{m}\cdot\mathrm{s}^{-1}$ & 1 & 2 & 3 & 4 & 5 \\
        & 位移/$\mathrm{m}$ & 1 & 2 & 3 & 4 & 5 \\
        \bottomrule
    \end{tabular}
    \caption{当 $c$ 与相对速度相关时,求解得到几个特殊时刻的浮子、振子位移和速度数据}
    \label{answer-1-2}
\end{table}



\subsection{问题二}

\subsubsection{输出功率的推导}

波浪能系统的输出功率来自于 PTO 的阻尼力做功。
在 $t$ 时刻,设 PTO 的阻尼力为 $F_\text{PTO}$,记瞬时输出功率为 $P(t)$,则
    \begin{align}
        & F_\text{PTO}=c|v_1-v_2| \\
        & P(t)=F_\text{PTO}\Delta v=c(v_1-v_2)^2
    \end{align}
其中 $v_1$、$v_2$ 是浮子和振子的速度。
得到瞬时功率之后,将其在一段时间内累加(积分)并取平均值,即可得到平均输出功率
\begin{equation}
    \bar{P}=\dfrac{1}{T}\int\limits_0^TP(t)\dif t\approx\dfrac{1}{T}\sum_{i=1}^n\dfrac{P(t_i)+P(t_{i+1})}{2}\Delta t
\end{equation}
结合问题一的方法计算离散的速度、位移与时间的关系,我们容易算出给定条件下的 PTO 输出功率。

\subsubsection{最优参数的选择}

本题需要找出在 (1) $c$ 为常数且 $c\in[0, \N{100000}]\,(\mathrm{N}\cdot\mathrm{s}/\mathrm{m})$ 和 (2) $c=C|v_1-v_2|^p$,其中 $C\in[0, \N{100000}]\,(\mathrm{N}\cdot(\mathrm{s}/\mathrm{m})^{p+1})$ 和 $p\in[0, 1]$ 为常数的两种情况下,平均输出功率 $\bar{P}$ 的最大值和相应的常数系数。
本问题使用的与问题一所不同的量如表 \ref{consts-2} 所示。
\begin{table}[htbp]
    \centering
    \begin{tabular}{cc}
        \toprule
        符号 & 量的值 \\
        \midrule
        $F$ & $f\cos(\omega t)$,其中 $f=\N{4890}\,\mathrm{N}$,$\omega=\N{2.2143}\,\mathrm{s}^{-1}$ \\
        $c_0$ & $\N{167.8395}\,\mathrm{N}\cdot\mathrm{s}/\mathrm{m}$ \\
        $m_0$ & $\N{1165.992}\,\mathrm{kg}$ \\
        \bottomrule
    \end{tabular}
    \caption{问题二与问题一不同的量}
    \label{consts-2}
\end{table}

以上面的条件设计程序输入计算机,我们很容易得到对于情况 1($c$ 为定值),PTO 系统输出功率与阻尼系数 $c$ 的关系如图所示。
当输出功率最大时,输出功率、阻尼系数如表 \ref{answer-2-1} 所示。
\begin{table}[htbp]
    \centering
    \begin{tabular}{cc}
        \toprule
        输出功率/W & 阻尼系数/$\mathrm{N}\cdot\mathrm{s}/\mathrm{m}$ \\
        \midrule
        1 & 2 \\
        \bottomrule
    \end{tabular}
    \caption{输出功率的最大值,以及此时的阻尼系数}
    \label{answer-2-1}
\end{table}

而对于情况 2($c$ 与相对速度有关),我们同样可以求出 PTO 系统输出功率与阻尼比例系数 $C$、幂指数 $p$ 的关系如图所示。
当输出功率最大时,对应的 $C$ 和 $p$ 的值如表 \ref{answer-2-2} 所示。

\begin{table}[htbp]
    \centering
    \begin{tabular}{ccc}
        \toprule
        输出功率/W & 比例系数/$\mathrm{N}\cdot(\mathrm{s}/\mathrm{m})^{p+1}$ & 幂指数 \\
        \midrule
        1 & 2 & 3 \\
        \bottomrule
    \end{tabular}
    \caption{输出功率的最大值,以及此时的阻尼系数}
    \label{answer-2-2}
\end{table}

\subsection{问题三}

\subsubsection{运动方程的建立}

与问题一相比,问题三加入了中轴的旋转,使得整个系统的运动变得复杂。
我们有必要建立新的细化的物理模型,从而列出新的运动方程。

本问题的模型中,外部浮子有两种不同类型的运动:一种是在竖直方向波浪激励力的作用下,在铅直方向进行的直线运动;另一种是在波浪激励力矩的影响下,绕某个轴进行的转动。
而在浮子的内部,振子由 PTO 和中轴所约束,PTO 和中轴的根部用一绞链与浮子相连结,亦绕绞链所在轴进行转动。
为了便于问题的分析,我们拟定下面的前提:
\begin{enumerate}
    \item 浮子的转动是绕其质心进行的定轴转动,振子的转动中心则是中轴与中轴底座绞接处。中轴底座、转轴等结构的厚度为零。
    \item 整个装置在水平方向没有速度,只做「垂荡」和「纵摇」运动。
    \item 题目中所提供的波浪激励力矩 $L\cos(\omega t)$ 可视为力偶矩,它不为浮子的平动提供加速度。
\end{enumerate}

我们以整个装置静止直立在水面上时,浮子的质心为原点,以来流指向方向为 $x$ 轴正方向,以铅直向上为 $z$ 轴正方向,建立空间直角坐标系 $O-xyz$。
在装置运行的某一时刻,在 $xOz$ 投影面上装置的运行状态如图 \ref{status} 所示。
\begin{figure}[htbp]
    \centering
    \includegraphics[width=6cm]{fig/problem-3-1.pdf}
    \caption{某时刻,$xOz$ 投影面上的装置运行状态}
    \label{status}
\end{figure}

在图中,$G$ 是浮子的质心。
由前文的前提我们可以知道,装置运行时,浮子不产生 $x$ 方向的水平位移,故 $G$ 在整个运动过程中都在 $z$ 轴上。
另一方面,浮子是一个对称旋转体,故 $G$ 亦始终位于浮子的旋转轴(图中的虚线)上。
$A$ 点是绞链所在的点,由题目条件知道绞接处也位于浮子的旋转轴上。
$P$ 点是振子质心所在的点,$AP$ 连线表示的就是 PTO 和中轴构成的结构。
考虑装置中所有的转动,我们记 $\theta$ 为浮子的纵摇角,记 $\gamma$ 为振子相对于浮子的纵摇角,角度以图上视角逆时针为正值。
同时,为了便于分析,我们记 $l$ 为弹簧在此时的长度,记 $d$ 为线段 $AG$ 的长度,即转轴绞接处与浮子质心的距离。

方便起见,我们设此时 $G$ 点的坐标为 $(0, 0, z_G)$,设此时 $P$ 点的坐标为 $(x_P, 0, z_P)$。
对振子进行受力分析。
振子在水平和竖直方向产生的加速度分别为 $\dfrac{\dif^2 x_P}{\dif t^2}$ 和 $\dfrac{\dif^2 z_P}{\dif t^2}$,而振子受到的外力有 PTO 装置(弹簧和阻尼器)的推拉力 $F_\text{PTO}$、振子自身重力 $mg$ 以及其受到中轴(视为轻杆)的支持力 $F_{AB}$。
记 $F_{AB}$ 在 $x$ 和 $z$ 方向的分力分别为 $F_{ABx}$ 和 $F_{ABz}$。
在 $x$ 和 $z$ 方向分别列出振子的运动学方程,有
\begin{align}
    m\cdot\dfrac{\dif^2 x_P}{\dif t^2} &= -F_\text{PTO}\sin(\theta+\gamma)+F_{ABx} \label{problem-3-eq1}\\
    m\cdot\dfrac{\dif^2 z_P}{\dif t^2} &= F_\text{PTO}\cos(\theta+\gamma)-mg+F_{ABz}
\end{align}

考虑振子的转动,设其在弹簧长度为 $l$ 时的转动惯量为 $I_B(l)$。
以逆时针为正方向。
记绞接处旋转阻尼器产生的旋转力矩为 $M_\text{PTO}$,则可以列出如下的转动运动学方程
\begin{equation}
    I_B(l)\cdot\dfrac{\dif^2(\theta+\gamma)}{\dif t^2}=M_\text{PTO}-F_{ABx}\cdot l\cos(\alpha+\gamma)-F_{ABz}\cdot l\sin(\theta+\gamma)
\end{equation}

再来分析浮子。
在水平方向,由前文的前提分析可知,浮子受到的合力为零。
而在这一方向能产生分力的,只有 $F_{ABx}$ 和 $F_\text{PTO}$ 的反作用力。
我们容易得到
\begin{equation}
    F_\text{PTO}\sin(\theta+\gamma)-F_{ABx}=0
\end{equation}

同时,在竖直方向,浮子受到波浪激励力 $F$、自身重力 $Mg$、PTO 反作用力 $F_\text{PTO}$ 的竖直分力、杆支持力的反作用力的水平竖直分力 $F_{ABy}$,以及浮力(静水恢复力) $F_\text{浮}$、附加惯性力 $m_0\dfrac{\dif^2z_G}{\dif t^2}$、兴波阻尼力 $c_0\dfrac{\dif z_G}{\dif t}$。
因此有
\begin{equation}
    M\cdot\dfrac{\dif^2 z_G}{\dif t^2}=F+F_\text{浮}-Mg-F_\text{PTO}\cos(\theta+\gamma)-F_{ABy}-m_0\dfrac{\dif^2z_G}{\dif t^2}-c_0\dfrac{\dif z_G}{\dif t}
\end{equation}

最后,我们分析浮子的转动过程。
记波浪激励力矩为 $M$,浮子绕过其质心的轴的转动惯量为 $I_A$,附加转动惯量为 $I_0$,兴波阻尼力矩系数为 $c_\text{rot}$,静水恢复力矩系数为 $M_\text{rec}$,有
\begin{equation}
    I_A\dfrac{\dif^2\theta}{\dif t^2}=M-M_\text{PTO}-F_{ABx}\cdot d\cos\theta-F_{ABz}\cdot d\sin\theta-I_0\dfrac{\dif^2\theta}{\dif t^2}-c_\text{rot}\dfrac{\dif\theta}{\dif t}-M_\text{rec}\theta \label{problem-3-eq6}
\end{equation}

式 \eqref{problem-3-eq1}—\eqref{problem-3-eq6} 构成一组微分方程组。
我们对其中感兴趣的量是 $\theta$、$\gamma$、$l$ 和 $z_G$,它们都是时间 $t$ 的函数。
下面,我们把上面方程中除了这 4 个函数以外的值全都进行转化,得到只与它们有关的一组方程。

\subsubsection{因变量的求值}

考虑弹簧和阻尼器构成的 PTO。
容易知道
\begin{equation}
    F_\text{PTO}=-k(l-l_0)-c\dfrac{\dif l}{\dif t}
\end{equation}

考虑转轴绞接处的旋转 PTO,记扭转弹簧的刚度系数为 $k_\text{rot}$,旋转阻尼器的阻尼系数为 $c_\text{rot}$,有
\begin{equation}
    M_\text{PTO}=-k_\text{rot}\gamma-c_\text{rot}\dfrac{\dif \gamma}{\dif t}
\end{equation}

考虑几何关系,有
\begin{align}
    x_P&=d\sin\theta-l\sin(\theta+\gamma) \\
    z_P&=z_G-d\cos\theta+l\cos(\theta+\gamma)
\end{align}

考虑浮子的质心 $G$。
显然浮子是由一个无底圆锥面、一个圆柱侧面和一个圆面构成的组合体,对此组合体的各个部分求质心,我们容易求得此组合体的质心距离底面长度为 $d_G=\N{1.45189}\,\mathrm{m}$。
再由 \eqref{floating-relationship},可以计算出浮子处于平衡位置时浮子排出海水的体积V排。
考虑几何关系可知,圆柱体在水中沿某一固定中轴线做摆动运动时,
其排出水的体积不随其母线与与中轴线夹角的改变而发生变化。
因此我们可以求得当圆柱体的重心沿固定中轴线移动到位置ZG'时,其排开水的体积为

V排'=V排-Pi*|ZG'-ZG|

考虑振子绕转轴转动的转动惯量,设转轴绞接处到振子底部圆心的距离为l。显然振子是一个圆柱体,可以将其看作是若干个圆心平面的集合。
而每一个圆形平面可以看作是若干条半径的集合,则可对该振子进行积分。

\int_a^bf(x)\mathrm{d}x

容易求得该振子绕转轴转动的转动惯量的表达式为

258.149 + 774.448 l + 1548.9 l^2

考虑浮子绕海水中绕重心所在中轴线做纵摇运动的转动惯量,易知该浮子可以看成是由一个无底圆锥面、一个圆柱侧面和一个圆面的组合体。
分三段对转动惯量进行积分求解,圆锥面可以看成是由若干个半径不同的无厚度圆环组成的曲面,对若干个圆环的转动惯量进行积分,即可求得圆锥面的转动惯量。
圆柱面可以看成是由若干个半径相同的无厚度圆环组成的曲面,底面可以看成是若干个半径组成的圆面,同样对二者进行积分,将三个积分结果相加,即可求得浮子的转动惯量为

5506.558




