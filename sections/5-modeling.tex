\section{模型建立与求解}

\subsection{问题一}

\subsubsection{模型的构建和分析}

在开始之前,我们需要先构建整个装置的物理模型。

当装置在水中静止时,对质量为 $m$ 的振子进行受力分析,以铅直向上为正方向,其受到的重力为 $-mg$,受到弹簧弹力为 $F_0$。
容易知道这两个力合力为零。若记此时振子位移为 0,记此时弹簧压缩量为 $x_0$,弹簧的胡克系数为 $k$,则有 

\begin{equation}
    kx_0^2-mg=0 
    \label{oscilator}
\end{equation}

考虑在 $t$ 时刻,浮子位移 $x_1$,振子位移 $x_2$。
记阻尼器的阻尼系数为 $c$,则

\begin{equation}
    m\dfrac{\dif^2x_2}{\dif t^2}=k(x_1-x_2+x_0)+c(\dfrac{\dif{x_1}}{\dif{t}}-\dfrac{\dif{x_2}}{\dif{t}})-mg
\end{equation}

而在此时,对浮子进行受力分析,记海水的兴波阻尼系数为 $c_0$,附加质量为 $m_0$,浮子质量为 $M$,波浪激励力为 $F$,有 

\begin{equation}
    M\dfrac{\dif^2x_1}{\dif t^2}=F+F_\text{浮}-Mg-k(x_1-x_2+x_0)-c(\dfrac{\dif{x_1}}{\dif{t}}-\dfrac{\dif{x_2}}{\dif{t}})-c_0\dfrac{\dif{x_1}}{\dif{t}}-m_0\dfrac{\dif^2 x_1}{\dif x_1^2}
\end{equation}

我们需要确定起始时装置的深度。
记装置静止时受到的总浮力为 $F_\text{浮总}$,水的密度为 $\rho$,装置在水中漂浮时排水体积为 $V_\text{排总}$,则有

\begin{align}
    & F_\text{浮总}=\rho gV_\text{排总} \\
    & (m+M)g=F_\text{浮总}
\end{align}

代入附件 4 的数据容易求得 $V_\text{排总}\approx7.121\,\mathrm{m}^3$。
另一方面,考虑浮子下方锥体的体积,有 $V_\text{锥}=\dfrac{1}3\uppi r^2h\approx0.837\,\mathrm{m}^3$。显然,水没过了整个锥体部分,水面在浮子的柱体部分。此时,浮子因在液面中上升下降所造成的浮力变化 $\Delta F_\text{浮}=\rho g\Delta V_{排}=\rho gA\Delta x_1$,其中 $\Delta x_1$ 为浮子没入液面深度的变化量,$A$ 是圆柱的横截面面积。
从而有

\begin{equation}
    F_\text{浮}=F_\text{浮总}+\Delta F_\text{浮}=F_\text{浮总}-\rho gAx_1
    \label{floating}
\end{equation}

联立 \eqref{oscilator}—\eqref{floating} 式,我们可以约去 $mg$、$Mg$、$F_\text{浮}$ 和 $x_0$ 几项,得到

\begin{equation}
    \left\{
    \begin{aligned}
        &  m\dfrac{\dif^2x_2}{\dif t^2}=k(x_1-x_2)+c(\dfrac{\dif{x_1}}{\dif{t}}-\dfrac{\dif{x_2}}{\dif{t}}) \\
        &  M\dfrac{\dif^2x_1}{\dif t^2}=F-k(x_1-x_2)-c(\dfrac{\dif{x_1}}{\dif{t}}-\dfrac{\dif{x_2}}{\dif{t}})-\rho gAx_1-c_0\dfrac{\dif{x_1}}{\dif{t}}-m_0\dfrac{\dif^2 x_1}{\dif t^2}
    \end{aligned}
    \right.
    \label{chui-dang}
\end{equation}

式 \eqref{chui-dang} 就是关于浮子和振子垂荡运动的微分方程组。
考虑装置启动前静止,$t=0$ 时,有初值

\begin{equation}
    \begin{aligned}
        & x_1=x_2=0 \\
        & \dfrac{\dif x_1}{\dif t}=\dfrac{\dif x_2}{\dif t}=0
    \end{aligned}
\end{equation}

求解这个初值问题,我们就能解出此题。