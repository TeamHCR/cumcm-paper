\section*{摘要}
\addcontentsline{toc}{section}{摘要}
随着化石矿物质能源的日益枯竭,新型能源如海洋波浪能已成为研究人员关注的焦点。
本文主要研究一种由浮子、振子等结构构成的波浪能装置,通过建立合理的数学模型,
推算其浮子和振子在不同情况下的摇荡位移和速度,并确定该系统在考虑不同因素的情况下的最大输出功率及相应的最优阻尼系数。

\textbf{对于问题一},我们通过对装置进行受力分析,列出与浮子、振子的位移有关的微分方程组,数值求解得出题目中所求的各个时刻的浮子与振子的垂荡位移和速度。
方程求解的结果已经存放到指定文件中。
其中,两种不同情况下 5 个特定时刻的状态如表 \ref{answer-1} 和表 \ref{answer-2} 所示:

\begin{table}[htbp]
    \centering
    \scriptsize
    \begin{minipage}[t]{0.48\textwidth}
        \centering
        \begin{tabular}{ccccccc}
            \toprule
            & & 10\,s & 20\,s & 40\,s & 60\,s & 100\,s \\
            \midrule
            \multirow{2}{*}{浮子} & 速度/$\mathrm{m}\cdot\mathrm{s}^{-1}$ & 1 & 2 & 3 & 4 & 5 \\
            & 位移/$\mathrm{m}$ & 1 & 2 & 3 & 4 & 5 \\
            \multirow{2}{*}{振子} & 速度/$\mathrm{m}\cdot\mathrm{s}^{-1}$ & 1 & 2 & 3 & 4 & 5 \\
            & 位移/$\mathrm{m}$ & 1 & 2 & 3 & 4 & 5 \\
            \bottomrule
        \end{tabular}
    \caption{(1) 中 5 个时刻的位移和速度}
    \label{answer-1}
    \end{minipage}
    \begin{minipage}[t]{0.48\textwidth}
        \centering
        \begin{tabular}{ccccccc}
            \toprule
            & & 10\,s & 20\,s & 40\,s & 60\,s & 100\,s \\
            \midrule
            \multirow{2}{*}{浮子} & 速度/$\mathrm{m}\cdot\mathrm{s}^{-1}$ & 1 & 2 & 3 & 4 & 5 \\
            & 位移/$\mathrm{m}$ & 1 & 2 & 3 & 4 & 5 \\
            \multirow{2}{*}{振子} & 速度/$\mathrm{m}\cdot\mathrm{s}^{-1}$ & 1 & 2 & 3 & 4 & 5 \\
            & 位移/$\mathrm{m}$ & 1 & 2 & 3 & 4 & 5 \\
            \bottomrule
        \end{tabular}
    \caption{(2) 中 5 个时刻的位移和速度}
    \label{answer-2}
    \end{minipage}
\end{table}

\textbf{对于问题二},
我们通过计算瞬时功率并将瞬时功率进行积分,求出一段时间内的总发电量,进而计算出平均功率。
我们将不同条件下的平均功率作为目标函数进行优化,得到最大输出功率及相应的最优阻尼系数,分别为 (1) 100\,W, 0.1 和 (2) 200\,W, 0.2。

\textbf{对于问题三},我们建立统一的直角坐标系,在坐标系中正交分析浮子和振子的受力和运动情况,列出微分方程并借助机器进行数值求解。
在 5 个特殊时刻的浮子和振子的摇荡数据如表 \ref{answer-3} 所示。
各速度的参考系请参见正文中的定义。

\begin{table}[htbp]
    \centering
    \scriptsize
    \begin{tabular}{ccccccccccccc}
        \toprule
        & & 10\,s & 20\,s & 40\,s & 60\,s & 100\,s & & 10\,s & 20\,s & 40\,s & 60\,s & 100\,s \\
        \midrule
        \multirow{2}{*}{浮子} & 速度/$\mathrm{m}\cdot\mathrm{s}^{-1}$ & \N{-0.36356} & \N{0.32232} & \N{0.25613} & \N{-0.02043} & \N{0.10731} & 角速度/$\mathrm{s}^{-1}$ & \N{-0.26764} & \N{0.31846} & \N{0.08744} & \N{0.16733} & \N{0.36698} \\
        & 位移/$\mathrm{m}$ & \N{-0.06645} & \N{0.13674} & \N{-0.05800} & \N{0.28778} & \N{0.30520} & 角位移/rad & \N{0.22206} & \N{0.82083} & \N{0.38582} & \N{0.92471} & \N{0.88672} \\  
        \multirow{2}{*}{振子} & 速度/$\mathrm{m}\cdot\mathrm{s}^{-1}$ & \N{-0.08109} & \N{0.10798} & \N{0.00638} & \N{0.07530} & \N{ 0.11954} & 角速度/$\mathrm{s}^{-1}$ & \N{-0.00610} & \N{0.01502} & \N{0.00135} & \N{0.012130} & \N{0.018767} \\
        & 位移/$\mathrm{m}$ & \N{0.20973} & \N{0.28602} & \N{0.20124} & \N{0.32874} & \N{0.30858} & 角位移/rad & \N{0.00334} & \N{0.02090} & \N{0.006036183} & \N{0.02598} & \N{0.02289} \\  
        \bottomrule
    \end{tabular}
    \caption{问题三中 5 个特殊时刻的位移、速度和角位移、角速度}
    \label{answer-3}
\end{table}

\textbf{对于问题四},我们采用和问题二相类似的方法,通过分组优化加快寻找最优解的速度,求得最大输出功率为 \N{12345}\,W,对应阻尼系数 $c=\N{1234}\,\mathrm{N}\cdot\mathrm{s}/\mathrm{m}$,旋转阻尼系数 $c_\text{rot}=\N{1234}\,\mathrm{N}\cdot\mathrm{m}\cdot\mathrm{s}$。

\vfill

\textbf{关键词}:波浪能发电 \hspace{1em} 摇荡运动 \hspace{1em} 数值模型 \hspace{1em} 水动力