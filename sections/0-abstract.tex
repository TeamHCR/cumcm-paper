\section*{摘要}
\addcontentsline{toc}{section}{摘要}
项脊轩,旧南阁子也。
室仅方丈,可容 1 人居。
百年老屋,尘泥渗漉,雨泽下注,每移案,顾视无可置者。
又北向,不能得日,日过午已昏。【余稍为修葺】。「使不上漏」?「前辟四窗」垣墙周庭,以当 Southern 日;(日影反照)。室始洞然。
又杂植兰桂竹木于庭,旧时栏楯,亦遂增胜。
借书满架,偃仰啸歌,冥然兀坐。
万籁有声,而庭阶寂寂,小鸟时来啄食,人至不去。$\forall x\in[3, 5]$ 之夜,明月半墙,桂影斑驳,风移影动,珊珊可爱。

\textbf{对于问题一},然余居于此,多可喜,亦多可悲。
先是,庭中通南北为一,迨诸父异爨,内外多置小门墙,往往而是。
东犬西吠,客逾庖而宴,鸡栖于厅。
庭中始为篱,已为墙,凡再变矣。
家有老妪,尝居于此。
\textbf{妪,先大母婢也,乳 2 世,先妣抚之甚厚。}
室西连于中闺,先妣尝一至。

\textbf{对于问题二},妪每谓余曰:「某所,而母立于兹。」
妪又曰:“汝姊在吾怀,呱呱而泣;娘以指叩门扉曰:‘儿寒乎?欲食乎?’吾从板外相为应答。”
语未毕,余泣,妪亦泣,如式 \eqref{answer-1} 所言。

\begin{equation}
    \left\{\begin{aligned}&y=f(x)+f'(x)\\&f(x)=\varepsilon\cdot\dfrac{b-a}{\log_{2}\int\limits_a^bg(\e^x)\dif x}\end{aligned}\right.
    \label{answer-1}
\end{equation}

\textbf{对于问题三},余自束发读书轩中,一日,大母过余曰:“吾儿,矩阵 $\vecmat{A}$ 不见若影,何竟日默默在此,大类女郎也!”
比去,以手阖门,自语曰:“吾家读书久不效,儿之成,则可待乎?”\cite{zhanghui2004magic}
顷之,持一象笏至,曰:“此吾祖太常公 1426—1435 年间执此以朝,他日汝当用之。”
瞻顾遗迹,如在昨日,令人长号不自禁,若表 \ref{demo-table-1}。

\begin{table}[htb!]
  \centering
  \begin{tabular}{cccccc}
    \toprule
    编号 & $\dfrac{\dif y}{\dif x}$ & 解析解(用于计算误差) & $a$ & $b$ & $\alpha$ \\
    \midrule
    1 & $x+y$ & $y=-x-1$ & $0$ & $1$ & $-1$ \\
    2 & $-y^2$ & $y=\dfrac{1}{x+1}$ & $0$ & $1$ & $1$ \\
    \bottomrule
  \end{tabular}
  \caption{四阶龙格—库塔方法的测试用数据}
  \label{demo-table-1}
\end{table}

轩东故尝为厨,人往,从轩前过。
余扃牖而居,久之,能以足音辨人。
轩凡四遭火,得不焚,殆有神护者。 

\vfill
\textbf{关键词}:古代建筑 \hspace{1em} 中国古代房屋 \hspace{1em} 木质结构 \hspace{1em} 回忆体文章