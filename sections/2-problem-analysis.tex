\section{问题分析}

披绣闼,俯雕甍。
山原旷其盈视,川泽纡其骇瞩。
闾阎扑地,钟鸣鼎食之家;舸舰迷津,青雀黄龙之舳。
云销雨霁,彩彻区明。
落霞与孤鹜齐飞\footnote{日藏唐抄本有“落霞与孤骛齐飞,秋水共长天一色”之句。},秋水共长天一色。
渔舟唱晚,响穷彭蠡之滨;雁阵惊寒,声断衡阳之浦。

\subsection{问题一的分析}

遥襟甫畅,逸兴遄飞。
爽籁发而清风生,纤歌凝而白云遏。
睢园绿竹,气凌彭泽之樽;邺水朱华,光照临川之笔。
四美具,二难并。
穷睇眄于中天,极娱游于暇日。
天高地迥,觉宇宙之无穷;兴尽悲来,识盈虚之有数。
望长安于日下,目吴会于云间。
地势极而南溟深,天柱高而北辰远。
关山难越,谁悲失路之人;萍水相逢,尽是他乡之客。
怀帝阍而不见,奉宣室以何年?

\subsection{问题二的分析}

嗟乎!时运不济,命途多舛。
冯唐易老,李广难封。
屈贾谊于长沙,非无圣主;窜梁鸿于海曲,岂乏明时?
所赖君子安贫,达人知命。
老当益壮,宁移白首之心;穷且益坚,不坠青云之志。
酌贪泉而觉爽,处涸辙以犹欢。
北海虽赊,扶摇可接;东隅已逝,桑榆非晚。
孟尝高洁,空馀报国之心;阮籍猖狂,岂效穷途之哭? 

\subsection{问题三的分析}

勃,三尺微命,一介书生,无路请缨,等终军之弱冠;有怀投笔,慕宗悫之长风。
舍簪笏于百龄,奉晨昏于万里。
非谢家之宝树,接孟氏之芳邻。
他日趋庭,叨陪鲤对;今兹捧袂,喜托龙门。
杨意不逢,抚凌云而自惜;锺期既遇,奏流水以何惭?

呜呼!
胜地不常,盛筵难再。
兰亭已矣,梓泽丘墟。
临别赠言,幸承恩于伟饯;登高作赋,是所望于群公。
敢竭鄙诚,恭疏短引。
一言均赋,四韵俱成。
请洒潘江,各倾陆海云尔。