\section{问题分析}

下面我们对各问题进行简要的分析,为进一步的模型建立做好准备。

\subsection{问题一的分析}

问题一描述了一个在波浪中做垂荡运动的浮子模型。
对浮子进行分析,其除了受到重力、浮力外,还直接受到波浪产生的激励力以及海水带来的附加惯性力、兴波阻尼力和静水恢复力。
而在内部,浮子还受到由理想弹簧和阻尼器构成的能量输出系统(简称 PTO)产生的拉(推)力和阻尼力。
同时,对振子进行分析,其受到的力则有重力和 PTO 的推(拉)力与阻尼力。
我们可以通过对浮子和振子进行竖直方向的受力分析,以浮子和振子的位移作为变量,列出它们的微分方程组,借助计算机进行求解,从而得到任意时刻浮子和振子的位移和速度。

在进行受力分析的过程中,为了使问题简化,我们可以将一些力进行合并、等效,从而大幅度降低分析和求解的难度。
在求解微分方程方面,我们使用 Wolfram Engine for Developers 软件\footnote{即 Wolfram Mathematica 软件的免费版本。}进行数值求解。

\subsection{问题二的分析}

问题二建立在问题一的基础之上。
对阻尼器进行分析,我们容易通过先前得到的浮子、振子位移和速度,得到任意时刻阻尼器上的相对速度,进而计算出阻尼器产生的阻尼力。
再对这些「瞬时功率」进行累加积分,容易计算出阻尼器在一段时间内的瞬时功率。
我们通过编写 Python 程序实现一种类「随机梯度下降」的算法,求解出瞬时功率在题目给定范围内的最优值。

\subsection{问题三的分析}

问题三在问题一的简单垂荡模型之上,加入了本质为绕轴旋转的纵摇运动。
浮子和振子的摆动角度作为新的变量被引入。
为了统一场景便于计算,我们建立一个统一的直角坐标系,将考虑到的各浮子和振子运动纳入其中表示。
我们在 $x$ 和 $z$ 两个方向对各种力进行分解求和,列出与浮子和振子的位移和角度有关的微分方程,再采用与问题一类似的方法进行分析求解。

\subsection{问题四的分析}

问题四与问题二类似,需要计算平均功率。
由于问题三已经计算出各个时刻的浮子、振子状态,我们也需要计算一系列瞬时功率,并进行累加积分,即可得到相应工况下的平均功率。
我们使用与问题二相似的方法,寻找最优的工作条件。